\section{Teori}
\subsection{Biofilm}

\subsection{Undervassakustikk}
\subsubsection*{Medisinsk bruk i dag}
\subsection{Kavitasjon}

Kavitasjon er definert som danninga av og aktiviteten til bobler i ei væske. Boblene er då anten midt i væska eller i grenseflata til andre objekt i eller ved væska.

Hydrodynamisk kavitasjon første gang skildra i 1895 av Barnaby og Thornycroft som fann ut at det negative trykket måtte vere større enn 6 og 3/4 "pounds per square inch". \cite{Young:1999}

Det fins fire forskjellige forhold der kavitasjon oppstår i ei væske.

\begin{itemize}
\item Gasskavitasjon - trykkreduksjon eller temperaturauke i ei gassboble i væska.
\item Dampkavitasjon - trykkreduksjon i ei dampboble i væska.
\item Degassing - Ei gassboble blir fylt med meir gass frå væska ved diffusjon.
\item Koking - ved høg temperatur blir væska omgjort til damp i bobler.
\end{itemize}

Ein kan også dele kavitasjon inn i dei ulike metodane som blir brukt for å få kavitasjon:

\begin{itemize}
\item Hydrodynamisk kavitasjon - kavitasjon i ei væske i rørsle på ein slik måte at det oppstår trykkvariasjonar i systemets geometri.
\item Akustisk kavitasjon - lydbølgjer i væska er årsak til trykkvariasjonar.
\item Optisk kavitasjon - høgintensitets lys (foton) revner væska
\item Partikkelkavitasjon - partiklar med høg fart, t.d. proton, revner væska.
\end{itemize}

Vi skal i hovudsak sjå på akustisk kavitasjon.

XXX 2.2 Bubble Dynamics:
Blake Treshold Pressure og Blake Radius.



\subsubsection{Akustisk Kavitasjon}
Ved akustisk kavitasjon varierer ein det statiske trykket sinusoidalt(?) i ei væske slik at kavitasjon oppstår på ein av to måtar. Stabil eller forbigåande kavitasjon. Stabile bobler endrar storleiken sin om ein likevekt over fleire periodar medan forbigåande kavitasjon tek mindre enn ein periode. Bobla doblar minst storleiken sin i løpet av denne perioden, men kan også bli endå større for så å kollapse sterkt. Boblene inneheld oftast ei blanding av luft og damp frå væska.

Dei akustiske eigenskapane til væska blir endra når den byrjar å kavitere. Mellom anna kan ein ikkje lenger rekne med ein fast akustisk impedans, men vi kan seie at den effektive akustiske resistansen og kompliansen aukar.

Effekten av ein sterk boblekollaps kan berre observerast i ein avstand som er mindre enn berre nokre få radiar av boblestorleiken, men effekten er ganske dramatisk. Kortvarig vil ein få eit svært høgt trykk og svært høg temperatur i området rundt bobla og kan føre til dramatiske effektar på omgivnaden.

Det er i tre situasjonar akustisk kavitasjon kan oppstå.
\begin{itemize}
\item Ei gassboble i væska som anten flyt rundt åleine.
\item På ein partikkel i væska som ikkje er gjennomvåt og difor har noko gass på overflata.
\item Ei overflate i oppbevaringsmediumet for væska der det er fanga ei gassboble. 
\end{itemize}

For at kavitasjonen i det heile kan starte er det viktig at lydtrykket er over eit visst nivå. Dette nivået kallar vi kavitasjonsterskelen.

\subsubsection{Kavitasjonsterskel}

Neppiras \cite{NeppirasThreshold} har skrive ein god artikkel om kavitasjonsterskel for akustisk kavitasjon. I denne artikkelen foreslår han relativt enkle likningar som er meir enn nøyaktige nok til bruk i industrielle og medisinske apparat. Han gjev følgjande likning for bobledynamikken ved eit akustisk trykk:

\begin{math}
\begin{matrix}
R \ddot{R} + \frac{3}{2}R^2 = \frac{1}{\rho} (P_L - P_\infty) \\
\\
\text{der } P_\infty = P_0 - P_A \sin \omega t \\
\\
\text{og } P_L = \left(P_0 +\dfrac{2\sigma}{R_0}\right) \left(\dfrac{R_0}{R}\right)^{3\gamma} - \dfrac{2\sigma}{R_0} - \dfrac{4\mu \dot{R}}{R}
\end{matrix}
\end{math}

XXX Kavitasjonsterskelen er gitt med ein fin figur for 20kHz på side 44 \cite{Young:1999} figur 3.5. Spørre om lov til å låne?

XXX Idé: kan ein nytte vatn med kolsyre for å lettare få kavitasjon?

\subsubsection{Viktige parametrar}

XXX Nemn viktige parametrar for kavitasjon?

\subsection{Måleteknikk}

\subsubsection{Kalibrering}

\subsection{Transduser}

XXX Skriv inn notatar om transduserteknologi

\subsection{Lydtransisjon mellom fleire medium (aka. vann-glass-vann-veien)}


\subsection{Observering av kavitasjon}

XXX Skriv om korleis ein kan observere kavitasjon (side 179 \cite{Young:1999}